\documentclass[a4paper,12pt]{article}


\usepackage[usenames,dvipsnames,svgnames,table]{xcolor}

\usepackage{amsmath, amssymb, arydshln} 
\usepackage{bm, booktabs}
\usepackage{cancel, caption, color}
\usepackage{datetime, dcolumn}  % Align table columns on decimal point
\usepackage{epsfig, epsf, enumitem}
\usepackage{fancyhdr}
\usepackage[T1]{fontenc}
\usepackage{graphicx, geometry}
\geometry{letterpaper, portrait, margin=1in}
\usepackage{hyperref}
\usepackage{ifthen}
\usepackage[utf8]{inputenc}
\usepackage{lscape, longtable}
\usepackage{multirow}
\usepackage{natbib}
\usepackage{pifont}
\usepackage{ragged2e}
%\newlist{thematic}{itemize}{8}
%\setlist[thematic]{label=$\square$}
\usepackage{subfigure}
\usepackage{sectsty}
\chapterfont{\color{blue}}                   
\sectionfont{\color{cyan}}                   
\subsectionfont{\color{ProcessBlue}}  
\usepackage{times, tabularx}
\usepackage{tcolorbox}
\usepackage{upgreek}
\usepackage{verbatim}
%\usepackage{tikz}

%% http://en.wikibooks.org/wiki/LaTeX/Colors



%%%%%%%%%%%%%%%%%%%%%%%%%%%%%%%%%%%%%%%%%%%
%       from Astro2020   ssb_190975.tex    template
%%%%%%%%%%%%%%%%%%%%%%%%%%%%%%%%%%%%%%%%%%%
\newcommand{\cmark}{\ding{51}}%
\newcommand{\xmark}{\ding{55}}%
\newcommand{\done}{\rlap{$\square$}{\raisebox{2pt}{\large\hspace{1pt}\cmark}}%
\hspace{-2.5pt}}
\newcommand{\wontfix}{\rlap{$\square$}{\large\hspace{1pt}\xmark}}

%%%%%%%%%%%%%%%%%%%%%%%%%%%%%%%%%%%%%%%%%%%
%      tcolorbox, tabularx  parameters
%%%%%%%%%%%%%%%%%%%%%%%%%%%%%%%%%%%%%%%%%%%

\tcbuselibrary{skins}
\newcolumntype{Y}{>{\raggedleft\arraybackslash}X}

\tcbset{tab1/.style={enhanced, fonttitle=\bfseries, fontupper=\normalsize\sffamily,
colback=yellow!10!white,
colframe=red!50!black,
colbacktitle=Cerulean!40!white,
coltitle=black,center title}
%subtitle style={boxrule=0.4pt, colback=yellow!50!red!25!white} 
}

%% To fix list things: 
\setitemize{noitemsep,topsep=0pt,parsep=0pt,partopsep=0pt,leftmargin=*}
%\renewcommand{\labelitemi}{\tiny$\blacksquare$}
\renewcommand{\thesection}{\alph{section}}



%%%%%%%%%%%%%%%%%%%%%%%%%%%%%%%%%%%%%%%%%%%
%       define Journal abbreviations      %
%%%%%%%%%%%%%%%%%%%%%%%%%%%%%%%%%%%%%%%%%%%
\def\nat{Nat} \def\apjl{ApJ~Lett.} \def\apj{ApJ}
\def\apjs{ApJS} \def\aj{AJ} \def\mnras{MNRAS}
\def\prd{Phys.~Rev.~D} \def\prl{Phys.~Rev.~Lett.}
\def\plb{Phys.~Lett.~B} \def\jhep{JHEP} \def\nar{NewAR}
\def\npbps{NUC.~Phys.~B~Proc.~Suppl.} \def\prep{Phys.~Rep.}
\def\pasp{PASP} \def\aap{Astron.~\&~Astrophys.} \def\araa{ARA\&A}
\def\jcap{\ref@jnl{J. Cosmology Astropart. Phys.}}%
\def\physrep{Phys.~Rep.}

\newcommand{\preep}[1]{{\tt #1} }

%%%%%%%%%%%%%%%%%%%%%%%%%%%%%%%%%%%%%%%%%%%%%%%%%%%%%
%              define symbols                       %
%%%%%%%%%%%%%%%%%%%%%%%%%%%%%%%%%%%%%%%%%%%%%%%%%%%%%
\def \Mpc {~{\rm Mpc} }
\def \Om {\Omega_0}
\def \Omb {\Omega_{\rm b}}
\def \Omcdm {\Omega_{\rm CDM}}
\def \Omlam {\Omega_{\Lambda}}
\def \Omm {\Omega_{\rm m}}
\def \ho {H_0}
\def \qo {q_0}
\def \lo {\lambda_0}
\def \kms {{\rm ~km~s}^{-1}}
\def \kmsmpc {{\rm ~km~s}^{-1}~{\rm Mpc}^{-1}}
\def \hmpc{~\;h^{-1}~{\rm Mpc}} 
\def \hkpc{\;h^{-1}{\rm kpc}} 
\def \hmpcb{h^{-1}{\rm Mpc}}
\def \dif {{\rm d}}
\def \mlim {m_{\rm l}}
\def \bj {b_{\rm J}}
\def \mb {M_{\rm b_{\rm J}}}
\def \mg {M_{\rm g}}
\def \qso {_{\rm QSO}}
\def \lrg {_{\rm LRG}}
\def \gal {_{\rm gal}}
\def \xibar {\bar{\xi}}
\def \xis{\xi(s)}
\def \xisp{\xi(\sigma, \pi)}
\def \Xisig{\Xi(\sigma)}
\def \xir{\xi(r)}
\def \max {_{\rm max}}
\def \gsim { \lower .75ex \hbox{$\sim$} \llap{\raise .27ex \hbox{$>$}} }
\def \lsim { \lower .75ex \hbox{$\sim$} \llap{\raise .27ex \hbox{$<$}} }
\def \deg {^{\circ}}
%\def \sqdeg {\rm deg^{-2}}
\def \deltac {\delta_{\rm c}}
\def \mmin {M_{\rm min}}
\def \mbh  {M_{\rm BH}}
\def \mdh  {M_{\rm DH}}
\def \msun {M_{\odot}}
\def \z {_{\rm z}}
\def \edd {_{\rm Edd}}
\def \lin {_{\rm lin}}
\def \nonlin {_{\rm non-lin}}
\def \wrms {\langle w_{\rm z}^2\rangle^{1/2}}
\def \dc {\delta_{\rm c}}
\def \wp {w_{p}(\sigma)}
\def \PwrSp {\mathcal{P}(k)}
\def \DelSq {$\Delta^{2}(k)$}
\def \WMAP {{\it WMAP \,}}
\def \cobe {{\it COBE }}
\def \COBE {{\it COBE \;}}
\def \HST  {{\it HST \,\,}}
\def \Spitzer  {{\it Spitzer \,}}
\def \ATLAS {VST-AA$\Omega$ {\it ATLAS} }
\def \BEST   {{\tt best} }
\def \TARGET {{\tt target} }
\def \TQSO   {{\tt TARGET\_QSO}}
\def \HIZ    {{\tt TARGET\_HIZ}}
\def \FIRST  {{\tt TARGET\_FIRST}}
\def \zc {z_{\rm c}}
\def \zcz {z_{\rm c,0}}

\newcommand{\ltsim}{\raisebox{-0.6ex}{$\,\stackrel
        {\raisebox{-.2ex}{$\textstyle <$}}{\sim}\,$}}
\newcommand{\gtsim}{\raisebox{-0.6ex}{$\,\stackrel
        {\raisebox{-.2ex}{$\textstyle >$}}{\sim}\,$}}
\newcommand{\simlt}{\raisebox{-0.6ex}{$\,\stackrel
        {\raisebox{-.2ex}{$\textstyle <$}}{\sim}\,$}}
\newcommand{\simgt}{\raisebox{-0.6ex}{$\,\stackrel
        {\raisebox{-.2ex}{$\textstyle >$}}{\sim}\,$}}



\newcommand{\Msun}{M_\odot}
\newcommand{\Lsun}{L_\odot}
\newcommand{\lsun}{L_\odot}
\newcommand{\Mdot}{\dot M}

\newcommand{\sqdeg}{deg$^{-2}$}
\newcommand{\hi}{H\,{\sc i}\ }
\newcommand{\lya}{Ly$\alpha$\ }
%\newcommand{\lya}{Ly\,$\alpha$\ }
\newcommand{\lyaf}{Ly\,$\alpha$\ forest}
%\newcommand{\eg}{e.g.~}
%\newcommand{\etal}{et~al.~}
\newcommand{\lyb}{Ly$\beta$\ }
\newcommand{\cii}{C\,{\sc ii}\ }
\newcommand{\ciii}{C\,{\sc iii}]\ }
\newcommand{\civ}{C\,{\sc iv}\ }
\newcommand{\SiII}{Si\,{\sc ii}\ }
\newcommand{\SiIV}{Si\,{\sc iv}\ }
\newcommand{\mgii}{Mg\,{\sc ii}\ }
\newcommand{\feii}{Fe\,{\sc ii}\ }
\newcommand{\feiii}{Fe\,{\sc iii}\ }
\newcommand{\caii}{Ca\,{\sc ii}\ }
\newcommand{\halpha}{H\,$\alpha$\ }
\newcommand{\hbeta}{H\,$\beta$\ }
\newcommand{\hgamma}{H\,$\gamma$\ }
\newcommand{\hdelta}{H\,$\delta$\ }
\newcommand{\oi}{[O\,{\sc i}]\ }
\newcommand{\oii}{[O\,{\sc ii}]\ }
\newcommand{\oiii}{[O\,{\sc iii}]\ }
\newcommand{\heii}{He\,{\sc ii}\ }
%\newcommand{\heii}{[He\,{\sc ii}]\ }
\newcommand{\nv}{N\,{\sc v}\ }
\newcommand{\nev}{Ne\,{\sc v}\ }
\newcommand{\neiii}{[Ne\,{\sc iii}]\ }
\newcommand{\alii}{Al\,{\sc ii}\ }
\newcommand{\aliii}{Al\,{\sc iii}\ }
\newcommand{\siiii}{Si\,{\sc iii}]\ }


% Add this line to include date functionality
\usepackage{datetime}

\begin{document}
\raggedright
\huge
U.K. Space Missions and Policy \linebreak
\vspace{48pt}


\textcolor{Cerulean}{
The UK’s future with ESA as seen through an STFC astronomy priority lens}  
\normalsize
\linebreak \linebreak 
\vspace{48pt}


\noindent \textbf{Thematic Areas} \\
Primary: U.K. Strategic Priority \\
Secondary: Astrophysics; Science mandatory programmes 
\linebreak
\vspace{24pt}
  
\textbf{Principal Author} \\
Nicholas P. Ross	
 \linebreak						
Niparo Ltd. 
 \linebreak
\href{mailto:nic@niparo.org}{nic@niparo.org}
 \linebreak

\vspace{24pt}
%\date{\today}
16 October 2024 \\
% \clearpage

\vspace{24pt}
\justify
\textbf{Abstract}
Astronomy and astrophysics research has long been a strength in the
UKs academic and research portfolio.  Astronomy and astrophysics
missions that are launched into space often have deep UK academic
involvement including NASA and ESA missions.  In particular, the UK
Space Agency (UKSA) is responsible for much of the UK astrophysics
space mission programme, while the science strategy, early technology,
data challenges and exploitation are all within the remit of Science
and Technology Facilities Council (STFC).  Here, we addres the very
specific issue of how the STFC astrophysics priorities align with the
ESA Science mandatory programme and specifically the ESA
\textit{Cosmic Vision: Space Science for Europe 2015–2025} campaign.



\pagebreak
\smallskip
\smallskip
\noindent
\vspace{-12pt}
{\bfseries \textsc{\textcolor{Cerulean}{Overview}}}
With the new UK Government, the incoming Budget and the European Space
Agency (ESA) holding the next Ministerial Council meeting in November
2025, now is a critical time to understand the scope of the U.K.'s
future with ESA in general and as seen through the lens of the Science
and Technology Facilities Council (STFC) in particular. Here we
suggest five STFC astrophysics priorities (given in no particular
order) and how ESA missions have recently, or are imminently coming
online to address these.

\smallskip
\smallskip
Direct influence is taken from the STFC Astronomy Advisory Panel
Roadmap 2022.\footnote{DOI: 10.48550/arXiv.2301.05457v1} The UK Space
Agency (UKSA) is responsible for much of the UK astrophysics space
mission programme, while the science strategy, early technology, data
challenges and exploitation are all within the remit of STFC.  In this
document we are ruthlessly narrow in scope, and note how STFC
astronomy priorities line-up with the ESA Science mandatory programme
and specifically the ESA \textit{Cosmic Vision: Space Science for
Europe 2015–2025} campaign.  Below we give a top-level view of the
five STFC astrophysics priorities.  Table~\ref{tab:ESA_missions} gives
the current mission portfolio for the Cosmic Vision campaign while
Table~\ref{tab:STFC_mapping} maps these priorities to the facilities
available to STFC.


\smallskip
\smallskip
{\bfseries \textsc{\textcolor{Cerulean}{Cosmology, Dark energy and Dark matter}}}
We do not know the nature of the vast majority of what constitutes the
Universe. 'Dark Energy' is thought to be causing the expansion of the
Universe to accelerate, while 'Dark Matter' is understood to be
fundamental to structure formation for galaxies but does not interact
with light or other electromagnetic radiation.  Although we know that
the quantity of the energy-density of the Universe is 26\% dark matter
and 68\% dark energy, the exact nature of both dark matter and dark
energy remains a mystery.

\smallskip
The objective of the ESA \textit{Euclid} mission is to better
understand dark energy and dark matter by accurately measuring the
accelerating expansion of the universe.  The recently selected
\textit{Analysis of Resolved Remnants of Accreted galaxies as a Key
Instrument for Halo Surveys} (ARRAKIHS) is a planned mission to study
the dark matter haloes of galaxies.




\smallskip
\smallskip
\noindent
{\bfseries \textsc{\textcolor{Cerulean}{The Gravitational Universe}}}
The first direct observation of gravitational waves made by the LIGO
gravitational wave detectors in the USA of the event GW150914 in 2015
forever changed the way we observe and listen to the
Universe.\footnote{The 2017 Nobel Prize in Physics was awarded to
Rainer Weiss, Kip Thorne and Barry Barish for their role in the direct
detection of gravitational waves.There is a strong case that
Prof. Ronald Drever, alma mater University of Glasgow would have won
the Nobel Prize in the place of Barry Barish had he not died before
the Nobel Committee made their decision.}  Gravitational waves were
predicted in 1916 by Albert Einstein on the basis of his general
theory of relativity as ripples in spacetime.  Gravitational waves
transport energy as gravitational radiation and are radiated by
massive objects whose motion involves changes in acceleration given
the system has some non-spherically symmetry.  Critically,
gravitational waves allow you to test theories of gravity and are not
diminished to nearly the same extent as EM radiation, experiencing
very little absorption even over very large distances. This will allow
studies of the extremely early Universe.

\smallskip 
The ESA Laser Interferometer Space Antenna (LISA) will be the first
space-based gravitational-wave observatory and it aims to detect and
measure gravitational waves from astronomical sources directly using
laser interferometry. LISA has a range of science goals and is able to
access a frequency of gravitational waves that is not possible on
Earth.


\smallskip
\smallskip
\noindent
{\bfseries \textsc{\textcolor{Cerulean}{The Early Universe: Probing Cosmic Dawn}}}
The Early Universe, usually defined as the epoch after cosmic
microwave background emission (around $\approx$370,000 years after the
Big Bang) to about 1 billion years after the Big Bang is the era when
the first stars and galaxies are able to form.\footnote{For some
perspective, imagine the Universe today is an 80 year old
person. 370,000 years after the Big Bang is the equivalent to seeing
the 1-day old baby photo of the elderly human. 1 billion years for the
Universe is then roughly a couple of months before the human's 6th
birthday.}  This is the epoch of the Universe where conditions become
suitable for the first stars and the first early galaxies to form.
However, the abundance of neutral hydrogen at this epoch means the
Universe remains relatively opaque and direct observations are
difficult.

\smallskip
\smallskip
Operating to 2$\upmu$m in the near-infrared, ESA \textit{Euclid}
mission will deliver information on the Early Universe and 'cosmic
dawn' when the first stars ignited and shone.


\smallskip
\smallskip
\noindent
{\bfseries \textsc{\textcolor{Cerulean}{Stars and Galaxies: interconnected and symbiotic systems}}}
The Universe is characterized by an enormous range of physical scales and hierarchy in structure, from
stars and planetary systems to galaxies and a cosmological web of complex filaments.
The quest to understand these interconnected and symbiotic systems, including
how stars and galaxies are formed, live and die remains a key scientific endeavour. 
%%

\smallskip
\textit{Euclid} is already delivering information on galaxy evolution
while ESA \textit{NewAthena} is an X-ray observatory that will address
a range of open scientific questions.  This includes: determining the
mechanisms regulating the cosmological co-evolution of accreting black
holes and their host galaxies; constraining the kinematics of hot gas
and metals in massive halos (galaxy clusters and groups); constraining
supernova explosion mechanisms and studying stellar-planet
interactions through the magnetic activity in exoplanet-hosting
systems.


\smallskip
\smallskip
\noindent
{\bfseries \textsc{\textcolor{Cerulean}{The Origin of Life}}}
``Are we alone in the Universe?'' remains one of, if not the most fundamental
questions that humans can ask. Is there
extra-terrestrial life in our cosmic neighbourhood? And if there is,
what form does it take?  Over the course of the last 30 or so years,
astronomers have confirmed over 5,500 exoplanets - planets orbiting
stars other than the Sun, with nearly 1,000 stellar systems being
multi-planetary.  The task at hand is further discovery, but also
characterization of exoplanets to understand in detail if there are
signatures of life.

\smallskip
ESA CHaracterising ExOPlanets Satellite (CHEOPS) is an ongoing missionto determine the size of known extrasolar planets, which will allow the estimation of their mass, density, composition and their formation. 
ESA PLAnetary Transits and Oscillations of stars (PLATO) will search
for exoplanets and measure stellar oscillations.  ESA Atmospheric
Remote-sensing Infrared Exoplanet Large-survey (ARIEL) is a space
observatory which will observe transits of nearby exoplanets.  The
primary science goal of the ESA Comet Interceptor mission is ``to
characterise, a dynamically-new comet, including its surface
composition, shape, structure, and the composition of its gas coma.''
ESA  Jupiter Icy Moons Explorer (JUICE) is an interplanetary spacecraft on its way to orbit and study three icy moons of Jupiter: Ganymede, Callisto, and Europa. These planetary-mass moons are planned to be studied because they are thought to have beneath their frozen surfaces significant bodies of liquid water. 

\begin{table}[h!]
  \centering
  \caption{Overview of ESA Missions and Their Scientific Themes}
  \label{tab:ESA_missions}
  \vspace{0.3cm}  
  \renewcommand{\arraystretch}{1.5}  
  \begin{tabular}{@{} p{2.5cm} >{\centering\arraybackslash}p{1.2cm} p{7.23cm} >{\raggedleft\arraybackslash}p{3.0cm} @{}}
    \toprule
    \textsc{Mission} & \textsc{Class} & \textsc{Science Theme} & \textsc{Launch} \\
    \midrule 
    CHEOPS            & S1 & Measure known exoplanets size  & 18 Dec 2019 \\
    Solar Orbiter     & M1 & Close-up observations of the Sun & 10 Feb 2020 \\
    \textit{Euclid}   & M2 & Dark energy and dark matter & 01 Jul 2023 \\
    JUICE             & L1 & Explore the Jupiter system & 14 Apr 2023 \\
    \hdashline
    SMILE             & S2 & Interaction between Earth's magnetosphere and solar wind & Late 2025 \\
    PLATO             & M3 & Search for exoplanets and measure stellar oscillations & 2026 \\
    ARIEL             & M4 & Nearby exoplanet chemical composition and physical conditions & 2029 \\
    Comet Interceptor & F1 & Study a long-period comet or an interstellar object & 2029 \\
    EnVision          & M5 & High-resolution radar mapping of the surface of Venus and atmospheric studies & 2031 \\
    ARRAKIHS         & F2 & Dwarf galaxies and stellar streams & Early 2030s \\
    NewAthena            & L2 & High-energy Universe & 2035 \\
    LISA              & L3 & Gravitational Universe & 2035 \\
    \bottomrule
  \end{tabular}
\end{table}


\vspace{8pt}
\begin{table}
  \begin{center}
    \renewcommand{\arraystretch}{1.3}  
    \caption{STFC Astrophysics Science Priorities and ESA Cosmic Vision Portfolio}
    \label{tab:STFC_mapping}
    \vspace{0.3cm}  
    \begin{tabular}{lr}
      \toprule
      \textsc{STFC Science Priority} & \textsc{ESA Mission}  \\
      \midrule
      Gravitational Universe    & LISA  \\
      \hdashline
      The Early Universe         & \textit{Euclid}     \\
      \hdashline
      \multirow{2}{*}{Dark Energy \& Dark Matter} & \textit{Euclid}      \\
                                                  &   ARRAKIHS \\
      \hdashline
      \multirow{3}{*}{Stars and Galaxies} & \textit{Euclid}  \\
                                &  NewAthena \\
                                & ARRAKIHS \\
      \hdashline
      \multirow{5}{*}{The Origin of Life}  &  CHEOPS \\
                                                             &  JUICE \\
                                                             &  Comet Interceptor \\
                                                             &  PLATO  \\
                                                             &  ARIEL  \\
      \bottomrule
    \end{tabular}
  \end{center}
\end{table}





\end{document}

