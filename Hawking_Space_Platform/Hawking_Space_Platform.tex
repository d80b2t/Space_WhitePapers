 \documentclass[a4paper,12pt]{article}



\usepackage[usenames,dvipsnames,svgnames,table]{xcolor}

\usepackage{amsmath, amssymb, arydshln} 
\usepackage{bm, booktabs}
\usepackage{cancel, caption, color}
\usepackage{datetime, dcolumn}  % Align table columns on decimal point
\usepackage{epsfig, epsf, enumitem}
\usepackage{fancyhdr}
\usepackage[T1]{fontenc}
\usepackage{graphicx, geometry}
\geometry{letterpaper, portrait, margin=1in}
\usepackage{hyperref}
\usepackage{ifthen}
\usepackage[utf8]{inputenc}
\usepackage{lscape, longtable}
\usepackage{multirow}
\usepackage{natbib}
\usepackage{pifont}
\usepackage{ragged2e}
%\newlist{thematic}{itemize}{8}
%\setlist[thematic]{label=$\square$}
\usepackage{subfigure}
\usepackage{sectsty}
\chapterfont{\color{blue}}                   
\sectionfont{\color{cyan}}                   
\subsectionfont{\color{ProcessBlue}}  
\usepackage{times, tabularx}
\usepackage{tcolorbox}
\usepackage{upgreek}
\usepackage{verbatim}
%\usepackage{tikz}

%% http://en.wikibooks.org/wiki/LaTeX/Colors



%%%%%%%%%%%%%%%%%%%%%%%%%%%%%%%%%%%%%%%%%%%
%       from Astro2020   ssb_190975.tex    template
%%%%%%%%%%%%%%%%%%%%%%%%%%%%%%%%%%%%%%%%%%%
\newcommand{\cmark}{\ding{51}}%
\newcommand{\xmark}{\ding{55}}%
\newcommand{\done}{\rlap{$\square$}{\raisebox{2pt}{\large\hspace{1pt}\cmark}}%
\hspace{-2.5pt}}
\newcommand{\wontfix}{\rlap{$\square$}{\large\hspace{1pt}\xmark}}

%%%%%%%%%%%%%%%%%%%%%%%%%%%%%%%%%%%%%%%%%%%
%      tcolorbox, tabularx  parameters
%%%%%%%%%%%%%%%%%%%%%%%%%%%%%%%%%%%%%%%%%%%

\tcbuselibrary{skins}
\newcolumntype{Y}{>{\raggedleft\arraybackslash}X}

\tcbset{tab1/.style={enhanced, fonttitle=\bfseries, fontupper=\normalsize\sffamily,
colback=yellow!10!white,
colframe=red!50!black,
colbacktitle=Cerulean!40!white,
coltitle=black,center title}
%subtitle style={boxrule=0.4pt, colback=yellow!50!red!25!white} 
}

%% To fix list things: 
\setitemize{noitemsep,topsep=0pt,parsep=0pt,partopsep=0pt,leftmargin=*}
%\renewcommand{\labelitemi}{\tiny$\blacksquare$}
\renewcommand{\thesection}{\alph{section}}



%%%%%%%%%%%%%%%%%%%%%%%%%%%%%%%%%%%%%%%%%%%
%       define Journal abbreviations      %
%%%%%%%%%%%%%%%%%%%%%%%%%%%%%%%%%%%%%%%%%%%
\def\nat{Nat} \def\apjl{ApJ~Lett.} \def\apj{ApJ}
\def\apjs{ApJS} \def\aj{AJ} \def\mnras{MNRAS}
\def\prd{Phys.~Rev.~D} \def\prl{Phys.~Rev.~Lett.}
\def\plb{Phys.~Lett.~B} \def\jhep{JHEP} \def\nar{NewAR}
\def\npbps{NUC.~Phys.~B~Proc.~Suppl.} \def\prep{Phys.~Rep.}
\def\pasp{PASP} \def\aap{Astron.~\&~Astrophys.} \def\araa{ARA\&A}
\def\jcap{\ref@jnl{J. Cosmology Astropart. Phys.}}%
\def\physrep{Phys.~Rep.}

\newcommand{\preep}[1]{{\tt #1} }

%%%%%%%%%%%%%%%%%%%%%%%%%%%%%%%%%%%%%%%%%%%%%%%%%%%%%
%              define symbols                       %
%%%%%%%%%%%%%%%%%%%%%%%%%%%%%%%%%%%%%%%%%%%%%%%%%%%%%
\def \Mpc {~{\rm Mpc} }
\def \Om {\Omega_0}
\def \Omb {\Omega_{\rm b}}
\def \Omcdm {\Omega_{\rm CDM}}
\def \Omlam {\Omega_{\Lambda}}
\def \Omm {\Omega_{\rm m}}
\def \ho {H_0}
\def \qo {q_0}
\def \lo {\lambda_0}
\def \kms {{\rm ~km~s}^{-1}}
\def \kmsmpc {{\rm ~km~s}^{-1}~{\rm Mpc}^{-1}}
\def \hmpc{~\;h^{-1}~{\rm Mpc}} 
\def \hkpc{\;h^{-1}{\rm kpc}} 
\def \hmpcb{h^{-1}{\rm Mpc}}
\def \dif {{\rm d}}
\def \mlim {m_{\rm l}}
\def \bj {b_{\rm J}}
\def \mb {M_{\rm b_{\rm J}}}
\def \mg {M_{\rm g}}
\def \qso {_{\rm QSO}}
\def \lrg {_{\rm LRG}}
\def \gal {_{\rm gal}}
\def \xibar {\bar{\xi}}
\def \xis{\xi(s)}
\def \xisp{\xi(\sigma, \pi)}
\def \Xisig{\Xi(\sigma)}
\def \xir{\xi(r)}
\def \max {_{\rm max}}
\def \gsim { \lower .75ex \hbox{$\sim$} \llap{\raise .27ex \hbox{$>$}} }
\def \lsim { \lower .75ex \hbox{$\sim$} \llap{\raise .27ex \hbox{$<$}} }
\def \deg {^{\circ}}
%\def \sqdeg {\rm deg^{-2}}
\def \deltac {\delta_{\rm c}}
\def \mmin {M_{\rm min}}
\def \mbh  {M_{\rm BH}}
\def \mdh  {M_{\rm DH}}
\def \msun {M_{\odot}}
\def \z {_{\rm z}}
\def \edd {_{\rm Edd}}
\def \lin {_{\rm lin}}
\def \nonlin {_{\rm non-lin}}
\def \wrms {\langle w_{\rm z}^2\rangle^{1/2}}
\def \dc {\delta_{\rm c}}
\def \wp {w_{p}(\sigma)}
\def \PwrSp {\mathcal{P}(k)}
\def \DelSq {$\Delta^{2}(k)$}
\def \WMAP {{\it WMAP \,}}
\def \cobe {{\it COBE }}
\def \COBE {{\it COBE \;}}
\def \HST  {{\it HST \,\,}}
\def \Spitzer  {{\it Spitzer \,}}
\def \ATLAS {VST-AA$\Omega$ {\it ATLAS} }
\def \BEST   {{\tt best} }
\def \TARGET {{\tt target} }
\def \TQSO   {{\tt TARGET\_QSO}}
\def \HIZ    {{\tt TARGET\_HIZ}}
\def \FIRST  {{\tt TARGET\_FIRST}}
\def \zc {z_{\rm c}}
\def \zcz {z_{\rm c,0}}

\newcommand{\ltsim}{\raisebox{-0.6ex}{$\,\stackrel
        {\raisebox{-.2ex}{$\textstyle <$}}{\sim}\,$}}
\newcommand{\gtsim}{\raisebox{-0.6ex}{$\,\stackrel
        {\raisebox{-.2ex}{$\textstyle >$}}{\sim}\,$}}
\newcommand{\simlt}{\raisebox{-0.6ex}{$\,\stackrel
        {\raisebox{-.2ex}{$\textstyle <$}}{\sim}\,$}}
\newcommand{\simgt}{\raisebox{-0.6ex}{$\,\stackrel
        {\raisebox{-.2ex}{$\textstyle >$}}{\sim}\,$}}



\newcommand{\Msun}{M_\odot}
\newcommand{\Lsun}{L_\odot}
\newcommand{\lsun}{L_\odot}
\newcommand{\Mdot}{\dot M}

\newcommand{\sqdeg}{deg$^{-2}$}
\newcommand{\hi}{H\,{\sc i}\ }
\newcommand{\lya}{Ly$\alpha$\ }
%\newcommand{\lya}{Ly\,$\alpha$\ }
\newcommand{\lyaf}{Ly\,$\alpha$\ forest}
%\newcommand{\eg}{e.g.~}
%\newcommand{\etal}{et~al.~}
\newcommand{\lyb}{Ly$\beta$\ }
\newcommand{\cii}{C\,{\sc ii}\ }
\newcommand{\ciii}{C\,{\sc iii}]\ }
\newcommand{\civ}{C\,{\sc iv}\ }
\newcommand{\SiII}{Si\,{\sc ii}\ }
\newcommand{\SiIV}{Si\,{\sc iv}\ }
\newcommand{\mgii}{Mg\,{\sc ii}\ }
\newcommand{\feii}{Fe\,{\sc ii}\ }
\newcommand{\feiii}{Fe\,{\sc iii}\ }
\newcommand{\caii}{Ca\,{\sc ii}\ }
\newcommand{\halpha}{H\,$\alpha$\ }
\newcommand{\hbeta}{H\,$\beta$\ }
\newcommand{\hgamma}{H\,$\gamma$\ }
\newcommand{\hdelta}{H\,$\delta$\ }
\newcommand{\oi}{[O\,{\sc i}]\ }
\newcommand{\oii}{[O\,{\sc ii}]\ }
\newcommand{\oiii}{[O\,{\sc iii}]\ }
\newcommand{\heii}{He\,{\sc ii}\ }
%\newcommand{\heii}{[He\,{\sc ii}]\ }
\newcommand{\nv}{N\,{\sc v}\ }
\newcommand{\nev}{Ne\,{\sc v}\ }
\newcommand{\neiii}{[Ne\,{\sc iii}]\ }
\newcommand{\alii}{Al\,{\sc ii}\ }
\newcommand{\aliii}{Al\,{\sc iii}\ }
\newcommand{\siiii}{Si\,{\sc iii}]\ }

 

\begin{document}
\raggedright
\huge
U.K. Space Missions and Policy \linebreak
\vspace{48pt}


\textcolor{Cerulean}{
The \textit{Hawking Space Platform}: A U.K. flagship mission to pioneer the circular space ecosystem}  
\normalsize
\linebreak \linebreak 
\vspace{48pt}


\noindent \textbf{Thematic Areas} \\
Primary: U.K. Strategic Priority \\
Secondary: Circular Economy; Sustainable Space 
\linebreak
\vspace{24pt}
  
\textbf{Principal Author} \\
Nicholas P. Ross	
 \linebreak						
Niparo Ltd. 
 \linebreak
\href{mailto:nic@niparo.org}{nic@niparo.org}
 \linebreak
 
%\textbf{Co-authors:} %% (names and institutions)
 % \linebreak
 % Roberto J. Assef (Universidad Diego Portales)  \linebreak
 % Matthew J. Graham (Caltech)  \linebreak 
 % J. Davy Kirkpatrick (Caltech/IPAC)  \linebreak

%\textbf{Date:} %% (names and institutions)
%\linebreak
13 October 2024

% \clearpage

\vspace{24pt}
\justify
\textbf{Abstract}
Missions and objects launched to Outer Space have traditionally been
bespoke and single-use.  Recently, the dramatic increase in cadence of
launches and mass-to-orbit means that certain orbits, in particulate
Low Earth Orbit (LEO; 200-2000km) are rapidly becoming not only
congested but are also at high risk for sites of collisions resulting in non-use.
As such, a new approach is needed in order to futher continue the
utilization of space for all of humankinds goals.  Here we present the
\textit{Hawking Space Platform}, a flagship mission for the UK space
sector, that will engage leading research institutes, Primes,
Start-up/SMEs, Scale-ups and Government into bold, concerted action
for space sustainability and the future of the circular economy in
space.
\textit{Hawking Space Platform} will be a modular spacecraft that will function as an observatory for astrophysics, a free-fall (``zero-g'') laboratory for a range of sciences and be a technology demonstrator for on-orbit circular economy practices. We motivate the \textit{Hawking Space Platform}, suggest key
mission parameters and requirements and argue that bold action is taken now by those interested in the
short-term and long-term benefits of outer space.




\pagebreak
\smallskip
\smallskip
\noindent
{\bfseries \textsc{\textcolor{Cerulean}{Overview}}} Numerous
authorities, including the UK Space Agency (UKSA) and the European
Space Agency (ESA) have identified space debris as an existential
threat to space operations, especially in Low Earth Orbit (LEO;
200-2000km).  Indeed, even in case of no further launches into orbit,
it is expected that collisions among the space debris objects already
present will lead to catastrophic collisions in LEO in the near
future.\footnote{ESA, Annual Space Environment Report, 2004,
GEN-DB-LOG-00288-OPS-SD; Figure 7.}  Based on these findings there is
a growing consensus that stricter space debris mitigation practices
need to be implemented globally, and, eventually, remediation will need to 
take place. 

\smallskip
\smallskip
\noindent
As such, the UK has been building leadership in space debris removal
and specifically Active Debris Removal (ADR) with the \textit{Clearing
the LEO Environment with Active Removal} (CLEAR) mission from
ClearSpace UK and the \textit{Cleaning Outer Space Mission through
Innovative Capture} (COSMIC) mission from Astroscale Ltd. set to
launch in 2026.\footnote{https://www.adsadvance.co.uk/astroscale-to-continue-uk-national-debris-removal-mission.html}

\smallskip
\smallskip
\noindent
While the ADR missions are a critical start to limiting the severe
impact of large space debris, the space sector is looking towards a
fully circular economy for space - a model of resource production and
consumption that involves reusing, repairing, refurbishing, and
recycling existing materials and products for as long as possible.
Indeed there is deep energy and fierce urgency in both the academical and industrial
spheres with a range of UK Space companies that are looking to
implement e.g. In-Orbit Servicing and Manufacturing
(IOSM)\footnote{https://www.ukspace.org/event/in-orbit-servicing-and-manufacturing-iosm-conference-2024; https://www.gov.uk/government/news/space-sustainability-conference-kicks-off-with-18-million-for-tech-innovation}.

\smallskip
\smallskip
\noindent
{\bfseries \textsc{\textcolor{Cerulean}{The Circular Economy in Astrophysics}}}
Interestingly, repairing and refurbishing space assets has a long
legacy and heritage in astrophysics.  Indeed, one of, if not the most
iconic space telescopes - the \textit{Hubble Space Telescope} - was
specifically designed to be repaired and refurbished over its mission
lifetime.  \textit{Hubble} had regular servicing and equipment upgrades while in orbit as well as orbital boosts.  Instruments and limited life items were designed as orbital replacement units with
five servicing missions flown by the NASA Space Shuttles, the last in May
2009. Some components from older instruments were reused in later instrument generations.\footnote{\href{https://en.wikipedia.org/wiki/Hubble\_Space\_Telescope\#List\_of\_Hubble\_instruments}{The first WFPC was dismantled, and some components were then re-used in WFC3.}
}

\smallskip \smallskip
\noindent
{\bfseries \textsc{\textcolor{Cerulean}{This Proposal}}}
Here, we propose 
%a Cross-Cluster Proof of Concept Call that will explore the proof of concept for 
the \textit{Hawking Space
Platform}. The \textit{Hawking Space Platform} will be a %large but
modular spacecraft that will function both as an observatory for
astrophysics and also as a free-fall (``zero-g'') laboratory for
Energy, Health and Life Sciences, Quantum, Digital, and Defence and Security  
investigations.  The \textit{Hawking Space
Platform} will be akin to the \textit{Hubble Space Telescope} in that
it will have instrument and laboratory bays where observations
and experiments can be carried out. Also like \textit{Hubble}, these
instruments will be designed to be replaced at a regular cadence,
making sure technology is continually bleeding-edge.
%%
The key mission parameters of the \textit{Hawking Space Platform} are given in Table~1.
%\ref{tab:HSP_mission_params} 

\smallskip \smallskip
\noindent
Unlike \textit{Hubble}, however, the \textit{Hawking Space Platform} will
have the instruments and laboratories upgraded via robotic
interactions (and not human intervention and Space Shuttle
infrastructure).  Designing this ability will enable a range of
technology demonstrators including, but not limited to: Rendezvous and
Proximity Operations (RPO); Advanced Guidance \& Targeting, and
Interface standardisation.  Obsolete \textit{Hawking
Space Platform} equipment would be recycled on orbit or, indeed one (or more) 
of the instrument bays in the \textit{Hawking} could be a recycling
centre. Technological risks can be explored  and retired for a range of
innovative technologies, while regulations can be stress-tested in a
scientific environment.

\smallskip \smallskip
\noindent
{\bfseries \textsc{\textcolor{Cerulean}{Hawking Space Platform in Astrophysics context}}}
NASA WISE, % (40 cm; 3.4, 4.6, 12 and 22$\mu$m),
NASA NEO Surveyor and % (50 cm; 4-5.2 and 6-10 $\mu$m) and
NASA SPHEREx %(20 cm; 0.75-5.0$\mu$m)
all have relatively modest primary mirror size (20-50cm) and operate
in the near-to-mid infrared (0.75-20$\upmu$m).\footnote{Noting the
NASA WISE mission became the NASA NEOWISE mission with only the 3.4,
4.6$\mu$m bands} 
A baseline for the observatory part of \textit{Hawking} could be close 
to the ESA Infrared Space Observatory (ISO). 
ISO was a 
%\euro480million 
space telescope with a 60cm primary mirror designed for wavelengths of 2.5 to 240$\upmu$m and
operated from 1995 to 1998.  The more precise design and optimization
for wavelength would be the focus of a Concept Study, though initially
wavelengths longer than $\sim$25$\upmu$m (the longest JWST MIRI
wavelength) which are often cryogenics-limited (see
Table~2) and close to 100\% absorbed by the Earth's atmosphere would be a natural waveband to initially explore.

%The technology drives the science; 
%Also a lab 
%Stone soup
%Observatory 
%Change instruments 
%Change primary mirror

\smallskip \smallskip
\noindent
{\bfseries \textsc{\textcolor{Cerulean}{Hawking Space Platform in budgetary and political context}}}
  We envisage the \textit{Hawking Space Telescope} to have a budget envelop of
greater than a NASA MIDEX mission and less than an ESA medium-class
(``M-class'') mission, putting this in the \pounds400-500~million range.
This would be, by some considerable distance, the most ambitious UK-led
civil space mission ever considered. However, this is directly comparable to
the investment from the UK Government in
OneWeb\footnote{https://www.gov.uk/government/news/uk-government-to-acquire-cutting-edge-satellite-network}
and significantly less than the Skynet family of military
communications satellites, so the UK does have precedent for this
ambition.  More importantly, the potential ROI could be truly significant and generational,  
immediately supercharging numerous scientific fields, sector innovation, giving the
UK launch sector a long-lived destination and be an industry-led, 
Government-backed space mission.  Universities,
\href{https://www.ukri.org/apply-for-funding/how-to-apply/check-if-you-are-eligible-for-research-and-innovation-funding/eligible-research-institutes/}{research
institutes},
\href{https://www.ukri.org/apply-for-funding/how-to-apply/check-if-you-are-eligible-for-research-and-innovation-funding/eligible-public-sector-research-establishments/#contents-list}{public
sector research establishments}, Primes, UKRI (including of course STFC), the
UKSA and ESA would all be involved.

\smallskip \smallskip
\noindent
\textit{In conclusion, the \textit{Hawking Space Platform} will be the
flagship mission for the UK space sector, that engages leading
research institutes, Primes, Start-up/SMEs, Scale-ups and Government
into bold, concerted action for space sustainability and the future of
the circular economy in space.}
  \vspace{-18pt}

%\smallskip \smallskip
%\noindent
%- Multi wavelength scope all in one?!?!
%- Robotics at Heriott watt 
%- First every sustainable space telescope 

%\vspace{-8pt}
  \begin{table}
\begin{center}
  \begin{tabular}{lrl}
%     \toprule
%    & Key Mission parameters of the \textit{Hawking Space Platform& \\
    \multicolumn{3}{c}{Table 1: Key Mission parameters of the \textit{Hawking Space Platform}} \\
    % \hline
    \midrule 
%    \hline
%    \multirow{3}{4em}{Multiple row} & cell2 & cell3 \\
    Mass Launch/Payload	             & 2,000/800 kg & ESA \textit{Euclid} as baseline \\
    Initital Launcher                &	Ariane 6 & $>$2000 kg payload \\
    Subsequent Launches     &  various & from e.g. UK spaceports \\
    Orbit & LEO & 200-2000km (Perigee and Apogee) \\
    Payload Bays  & several & for Observatory instruments and laboratories \\
    Mass equivalent reused & $>$95\% & Primary Mission Requirement \\
%        Launch Date                &	2027 & \multirow{2}{flying concurrently with ESA \textit{Euclid}, JWST and NASA Nancy Roman, SPHEREx } \\
%        & & \\
    \bottomrule
% \hline
 %\hline
  \end{tabular}
  %\caption{}%Key Mission parameters of the \textit{Hawking Space Platform}}
  %\label{tab:HSP_mission_params}
  \end{center}
  \end{table}
  \vspace{-18pt}
  
  \begin{table}
\begin{center}
  \begin{tabular}{l l l}
    %\toprule
        \multicolumn{3}{c}{Table 2: List of selected astrophysics missions and reason for End of Mission} \\
    \midrule
    Astrophysics Mission   & Lifetime  & End of Mission cause\\
    % \hline
    \midrule
    ESA/NASA \textit{Hubble Space Telescope}   &         1990-ongoing  &  Likely re-entry into Earth's atmosphere\\
    ESA       ISO	                                                   &         1995-1998       &  Liquid helium depletion \\
    ESA    \textit{Herschel} Space Observatory     &         2009-2013	& Liquid helium depletion \\
    ESA     \textit{Planck}	                                   &         2009-2013	& Liquid helium depletion \\ %$^{1}$  \\
    NASA  \textit{Compton} GRO                         &  1991-2000        & Gyroscope failure and de-orbit \\ %$^{2}$ \\
    NASA  \textit{Kepler}                                              & 2009-2018 &  Reaction Control System fuel depleted \\
    NASA  \textit{Spitzer Space Telescope} &2003-2009 & Liquid helium depletion \\
    NASA WISE                                        &2009-2012 & Hydrogen coolant depletion \\
    \bottomrule
      \end{tabular}
%    \caption{
 %   $^{1}$For the High Frequency Instrument (HFI). 
  %  $^{2}$This de-orbit was NASA's first intentional controlled de-orbit of a satellite.}
\label{tab:space_missions}
  \end{center}
  \end{table}

    \vspace{-38pt}


    
\end{document}

